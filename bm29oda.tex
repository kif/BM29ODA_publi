%------------------------------------------------------------------------------
% Template file for the submission of papers to IUCr journals in LaTeX2e
% using the iucr document class
% Copyright 1999-2013 International Union of Crystallography
% Version 1.6 (28 March 2013)
%------------------------------------------------------------------------------

\documentclass[preprint]{iucr}              % DO NOT DELETE THIS LINE

     %-------------------------------------------------------------------------
     % Information about journal to which submitted
     %-------------------------------------------------------------------------
     \journalcode{J}              % Indicate the journal to which submitted
                                  %   A - Acta Crystallographica Section A
                                  %   B - Acta Crystallographica Section B
                                  %   C - Acta Crystallographica Section C
                                  %   D - Acta Crystallographica Section D
                                  %   E - Acta Crystallographica Section E
                                  %   F - Acta Crystallographica Section F
                                  %   J - Journal of Applied Crystallography
                                  %   M - IUCrJ
                                  %   S - Journal of Synchrotron Radiation

\begin{document}                  % DO NOT DELETE THIS LINE

     %-------------------------------------------------------------------------
     % The introductory (header) part of the paper
     %-------------------------------------------------------------------------

     % The title of the paper. Use \shorttitle to indicate an abbreviated title
     % for use in running heads (you will need to uncomment it).

\title{Online data analysis at ESRF BiosSaxs beamline}
%\shorttitle{Short Title}

     % Authors' names and addresses. Use \cauthor for the main (contact) author.
     % Use \author for all other authors. Use \aff for authors' affiliations.
     % Use lower-case letters in square brackets to link authors to their
     % affiliations; if there is only one affiliation address, remove the [a].

\cauthor[a]{Jérôme}{Kieffer}{jerome.kieffer@esrf.fr}
\author[b]{Adam}{Round}
\author[a]{Petra}{Pernot}
\author[a]{Martha}{Brennich}
\author[a]{Alejandro}{De Maria Antolinos}

\aff[a]{ESRF Grenoble TODO \country{France}}
\aff[b]{EMBL Grenoble TODO \country{France}}

     % Use \shortauthor to indicate an abbreviated author list for use in
     % running heads (you will need to uncomment it).

\shortauthor{Kieffer et all.}

     % Use \vita if required to give biographical details (for authors of
     % invited review papers only). Uncomment it.

%\vita{Author's biography}

     % Keywords (required for Journal of Synchrotron Radiation only)
     % Use the \keyword macro for each word or phrase, e.g. 
     % \keyword{X-ray diffraction}\keyword{muscle}

\keyword{Online data-analysis, solution scattering, protein}

     % PDB and NDB reference codes for structures referenced in the article and
     % deposited with the Protein Data Bank and Nucleic Acids Database (Acta
     % Crystallographica Section D). Repeat for each separate structure e.g
     % \PDBref[dethiobiotin synthetase]{1byi} \NDBref[d(G$_4$CGC$_4$)]{ad0002}

%\PDBref[optional name]{refcode}
%\NDBref[optional name]{refcode}

\maketitle                        % DO NOT DELETE THIS LINE

\begin{synopsis}
Low-latency data reduction and real-time feed back to the users 
\end{synopsis}

\begin{abstract}
Abstract+Introduction (with motivation) -> Adam
Saxs applies to proteins blabla
high throughput blabla
need automatic data treatement blabla
\end{abstract}


     %-------------------------------------------------------------------------
     % The main body of the paper
     %-------------------------------------------------------------------------
     % Now enter the text of the document in multiple \section's, \subsection's
     % and \subsubsection's as required.

\section{Introduction}

Adam ?
Text text text text text text text text text text text text text text
text text text text text text text.



\section{Online data analysis}

Online data analysis is needed on highly automated beamlines where the
acquisition of a sample lasts for less than a second and there is virtually no
dead-time between samples.

\subsection{Data analysis server}


\section{Data Acquisition using the sample changer}

The sample changer allows the 


\subsection{Azimuthal integration pipeline}

This pipeline is triggered 

Based on FabIO\cite{fabio} for image reading and pyFAI\cite{pyFAI} for azimuthal
integraton \ldots

\subsection{curve merging}


\subsection{\textem{ab-initio} reconstruction pipeline}

Text text text text text text text text text text text text text text
text text text text text text text.

\section{HPLC mode}

A couple of specific pipeline have been developed more recently when the
sample-changer is replaced by a steric exclusion column based  chromatographic
setup (hereafter named HPLC-mode).
In HPLC-mode the eluate of the column is directly connected to the capillary
and the buffer flows though continuously. 2D-diffraction frames are continuously
taken at a frequency around 1 Hz (the maximum achievable speed with the current
system is 30 Hz for the detector and around 10 Hz for the data-analysis
pipeline).
\subsection{HPLC frame}

This pipeline, triggered for every-frame, is similar to the azimuthal
integration pipeline used in sample-changer-mode as it performs the regrouping
of the image and provides the 1D curve.
In addition, this plugin is in charge of defining the reference buffer by
comparing any frame to the first one, assuming the eluate of the first frame
contains pure buffer (using datcmp from the ATSAS package).
Very similar frames (based on a threshold) are merged to the first one to define
the averaged buffer of the experiment.

Any frame very dissimilar (there is a second threshold) to the first/buffer
(check?) is considered as a sample frame hence the buffer signal is subtracted
and Guinier analysis is performed, providing a radius of giration and an
intensity of scattering at q=0.
 
\subsection{HPLC flush}
When the HPLC experiment ends (or is aborted) a specific plugin is launched to
finish the processing of the experiment:
* Merge all 1D diffraction patterns into a 2D dataset containing scattering
intensity in function of scattering vector and time.
* locate peaks, corresponding to the maximum of I0 in Guinier analysis
* merge curves from the same sample to enhance the statistic. All eluate curves
located around a local maxima of I0 with similar Rg are considered
containing the same compound and hence merged togeather.

All those results are saved into a single HDF5 which is uploaded to the ISPyBB
database at the end of the processing.

A figure describing the results would be welcome ! 

\section{Offline data analysis}
Describes reprocess mode &
explains why cannot be distributed (licensing, difficulty to install)
 
\section{Hardware used}
Two computers with GPU computing capabilities (Nvidia Quadro 4000) are dedicated
for online data analysis on the biosaxs bemline, they are independent and
feature the same software installation hence providing redondancy (if one
fails).
As the \textem{ab initio} reconstruction pipeline lasts of dozens of minutes
(due to the execution of DAMMIN), it is run on the most powerful computer while 
the azimuthal integration pipeline, which needs the lowest latency, is run on
the other computer.



\section{Conclusion}


\ack{Acknowledgements}
Ricardo Fernandes & Thomas Boeglin: original reprocess 
Peter Boesecke: original azimuthal integration procedure
Irakli for the original ab-initio pipeline 
Al & Dmitry for the atsas package and support 
Staffan Ohlsson & Matias Guijarro: bliss contact

\bibliographystyle{iucr}
\bibliography{biblio}

% \reference{Author, A. \& Author, B. (1984). \emph{Journal} \textbf{Vol}, 
% first page--last page.}
% \end{references}
% 
%      %-------------------------------------------------------------------------
%      % TABLES AND FIGURES SHOULD BE INSERTED AFTER THE MAIN BODY OF THE TEXT
%      %-------------------------------------------------------------------------
% 
%      % Simple tables should use the tabular environment according to this
%      % model
% 
% \begin{table}
% \caption{Caption to table}
% \begin{tabular}{llcr}      % Alignment for each cell: l=left, c=center, r=right
%  HEADING    & FOR        & EACH       & COLUMN     \\
% \hline
%  entry      & entry      & entry      & entry      \\
%  entry      & entry      & entry      & entry      \\
%  entry      & entry      & entry      & entry      \\
% \end{tabular}
% \end{table}
% 
%      % Postscript figures can be included with multiple figure blocks
% 
% \begin{figure}
% \caption{Caption describing figure.}
% \includegraphics{fig1.ps}
% \end{figure}
% 

\end{document}                
